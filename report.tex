\documentclass[a4paper]{article}
\usepackage{vntex}
%\usepackage[english,vietnam]{babel}
%\usepackage[utf8]{inputenc}

%\usepackage[utf8]{inputenc}
%\usepackage[francais]{babel}
\usepackage{a4wide,amssymb,epsfig,latexsym,multicol,array,hhline,fancyhdr}
\usepackage{booktabs}
\usepackage{amsmath}
\usepackage{lastpage}
\usepackage[lined,boxed,commentsnumbered]{algorithm2e}
\usepackage{enumerate}
\usepackage{color}
\usepackage{graphicx}							% Standard graphics package
\usepackage{array}
\usepackage{tabularx, caption}
\usepackage{multirow}
\usepackage[framemethod=tikz]{mdframed}% For highlighting paragraph backgrounds
\usepackage{multicol}
\usepackage{rotating}
\usepackage{graphics}
\usepackage{geometry}
\usepackage{setspace}
\usepackage{epsfig}
\usepackage{tikz}
\usepackage{listings}
\usetikzlibrary{arrows,snakes,backgrounds}
\usepackage{hyperref}
\hypersetup{urlcolor=blue,linkcolor=black,citecolor=black,colorlinks=true} 
%\usepackage{pstcol} 								% PSTricks with the standard color package

\newtheorem{theorem}{{\bf Định lý}}
\newtheorem{property}{{\bf Tính chất}}
\newtheorem{proposition}{{\bf Mệnh đề}}
\newtheorem{corollary}[proposition]{{\bf Hệ quả}}
\newtheorem{lemma}[proposition]{{\bf Bổ đề}}

\everymath{\color{blue}}
%\usepackage{fancyhdr}
\setlength{\headheight}{40pt}
\pagestyle{fancy}
\fancyhead{} % clear all header fields
\fancyhead[L]{
 \begin{tabular}{rl}
    \begin{picture}(25,15)(0,0)
    \put(0,-8){\includegraphics[width=8mm, height=8mm]{logoITSGUsmall.png}}
    %\put(0,-8){\epsfig{width=10mm,figure=hcmut.eps}}
   \end{picture}&
	%\includegraphics[width=8mm, height=8mm]{hcmut.png} & %
	\begin{tabular}{l}
		\textbf{\bf \ttfamily Trường Đại học Sài Gòn}\\
		\textbf{\bf \ttfamily Khoa Công Nghệ Thông Tin}
	\end{tabular} 	
 \end{tabular}
}
\fancyhead[R]{
	\begin{tabular}{l}
		\tiny \bf \\
		\tiny \bf 
	\end{tabular}  }
\fancyfoot{} % clear all footer fields
\fancyfoot[L]{\scriptsize \ttfamily Bài tập lớn môn Phát triển phần mềm mã nguồn mở - Niên khóa 2023-2024}
\fancyfoot[R]{\scriptsize \ttfamily Trang {\thepage}/\pageref{LastPage}}
\renewcommand{\headrulewidth}{0.3pt}
\renewcommand{\footrulewidth}{0.3pt}


%%%
\setcounter{secnumdepth}{4}
\setcounter{tocdepth}{3}
\makeatletter
\newcounter {subsubsubsection}[subsubsection]
\renewcommand\thesubsubsubsection{\thesubsubsection .\@alph\c@subsubsubsection}
\newcommand\subsubsubsection{\@startsection{subsubsubsection}{4}{\z@}%
                                     {-3.25ex\@plus -1ex \@minus -.2ex}%
                                     {1.5ex \@plus .2ex}%
                                     {\normalfont\normalsize\bfseries}}
\newcommand*\l@subsubsubsection{\@dottedtocline{3}{10.0em}{4.1em}}
\newcommand*{\subsubsubsectionmark}[1]{}
\makeatother

\definecolor{dkgreen}{rgb}{0,0.6,0}
\definecolor{gray}{rgb}{0.5,0.5,0.5}
\definecolor{mauve}{rgb}{0.58,0,0.82}

\lstset{frame=tb,
	language=Matlab,
	aboveskip=3mm,
	belowskip=3mm,
	showstringspaces=false,
	columns=flexible,
	basicstyle={\small\ttfamily},
	numbers=none,
	numberstyle=\tiny\color{gray},
	keywordstyle=\color{blue},
	commentstyle=\color{dkgreen},
	stringstyle=\color{mauve},
	breaklines=true,
	breakatwhitespace=true,
	tabsize=3,
	numbers=left,
	stepnumber=1,
	numbersep=1pt,    
	firstnumber=1,
	numberfirstline=true
}

\begin{document}

\begin{titlepage}
	\begin{center}
		TRƯỜNG ĐẠI HỌC SÀI GÒN \\
		KHOA CÔNG NGHỆ THÔNG TIN
	\end{center}
	\vspace{1cm}

	\begin{figure}[h!]
		\begin{center}
			\includegraphics[width=3cm]{logoITSGU.png}
		\end{center}
	\end{figure}

	\vspace{1cm}


	\begin{center}
		\begin{tabular}{c}
			\multicolumn{1}{l}{\textbf{{\Large PHÁT TRIỂN PHẦN MỀM MÃ NGUỒN MỞ}}} \\
			~~                                                                    \\
			\hline
			\\
			\multicolumn{1}{l}{\textbf{{\Large Phát triển}}}                      \\
			\\

			\textbf{{\Huge Ứng dụng nghe nhạc trên PYTHON}}                       \\
			\\
			\hline
		\end{tabular}
	\end{center}

	\vspace{3cm}

	\begin{table}[h]
		\begin{tabular}{rrl}
			\hspace{5 cm} & GVHD: & Từ Lãng Phiêu                     \\
			              & SV:   & Nguyễn Anh Danh - 3121410103      \\
			              &       & Phan Duy - 3121410003             \\
			              &       & Văn Phú Hiếu - 3121410201         \\
			              &       & Đỗ Nguyễn Hoàng Tuấn - 3121410554 \\
			% & & SV3 - MSSV \\
			% & & SV4 - MSSV\\
		\end{tabular}
		\vspace{1.5 cm}
	\end{table}

	\begin{center}

		{\footnotesize TP. HỒ CHÍ MINH, THÁNG 5/2024}
	\end{center}
\end{titlepage}


\thispagestyle{empty}
\newpage
\begin{center}
	\section*{Nhận xét, đánh giá của giảng viên}
\end{center}
\begin{flushleft}
	\dotfill
\end{flushleft}
\begin{flushleft}
	\dotfill
\end{flushleft}
\begin{flushleft}
	\dotfill
\end{flushleft}
\begin{flushleft}
	\dotfill
\end{flushleft}
\begin{flushleft}
	\dotfill
\end{flushleft}
\begin{flushleft}
	\dotfill
\end{flushleft}
\begin{flushleft}
	\dotfill
\end{flushleft}
\begin{flushleft}
	\dotfill
\end{flushleft}
\begin{flushleft}
	\dotfill
\end{flushleft}
\begin{flushleft}
	\dotfill
\end{flushleft}
\begin{flushleft}
	\dotfill
\end{flushleft}
\begin{flushleft}
	\dotfill
\end{flushleft}
\begin{flushleft}
	\dotfill
\end{flushleft}
\begin{flushleft}
	\dotfill
\end{flushleft}
\begin{flushleft}
	\dotfill
\end{flushleft}
\begin{flushleft}
	\dotfill
\end{flushleft}
\begin{flushleft}
	\dotfill
\end{flushleft}
\begin{flushleft}
	\dotfill
\end{flushleft}
\begin{flushleft}
	\dotfill
\end{flushleft}
\begin{flushleft}
	\dotfill
\end{flushleft}
\begin{flushleft}
	\dotfill
\end{flushleft}
\begin{flushleft}
	\dotfill
\end{flushleft}
\begin{flushleft}
	\dotfill
\end{flushleft}
\begin{flushleft}
	\dotfill
\end{flushleft}
\begin{flushleft}
	\dotfill
\end{flushleft}
\begin{flushleft}
	\dotfill
\end{flushleft}
%%%%%%%%%%%%%%%%%%%%%%%%%%%%%%%%%


%%%%%%%%%%%%%%%%%%%%%%%%%%%%%%%%%
\newpage
\begin{center}
	\section*{Lời cảm ơn}
\end{center}
\begin{flushleft}
	Chúng em xin gửi lời cảm ơn chân thành nhất đối với các thầy cô ở khoa Công Nghệ Thông Tin, trường Đại học Sài Gòn đã tạo điều kiện cho chúng em tiếp cận và tìm hiểu để hoàn thành đồ án môn học lần này. Và chúng em cũng xin chân thành cảm ơn thầy Từ Lãng Phiêu giáo viên giảng dạy đã nhiệt tình hướng dẫn chúng em hoàn thành đồ án lần này.
	Trong quá trình thực hiện nghiên cứu và thực hiện làm báo cáo đồ án, do kinh nghiệm thực tế chưa được nhiều, nên bài báo cáo của chúng em có thể vẫn còn những thiếu sót và chưa được hoàn chỉnh nên mong rằng chúng em sẽ nhận được những đóng góp ý kiến đóng góp bổ ích từ thầy để chúng em có thể khắc phục cho những bài báo cáo sau.

\end{flushleft}
\begin{flushright}
	\text{Chúng em xin trân trọng cảm ơn thầy!}
\end{flushright}

\newpage
\tableofcontents
\newpage

%%%%%%%%%%%%%%%%%%%%%%%%%%%%%%%%%


%%%%%%%%%%%%%%%%%%%%%%%%%%%%%%%%%
\section{Phần 1: Mở đầu}
\subsection{Lý do chọn đề tài}

Công nghệ thông tin ngày càng trở thành một phần không thể thiếu trong cuộc sống hiện đại, và ngôn ngữ lập trình Python đã và đang đóng một vai trò quan trọng trong việc phát triển các ứng dụng công nghệ thông tin. Python với cấu trúc rõ ràng, dễ đọc và dễ học, đã trở thành một trong những lựa chọn hàng đầu cho nhiều lập trình viên trên toàn thế giới.

Âm nhạc là một phần quan trọng của cuộc sống, mang lại niềm vui, sự thư giãn và là nguồn cảm hứng cho con người. Với sự phát triển của công nghệ, việc nghe nhạc đã trở nên dễ dàng hơn bao giờ hết. Tuy nhiên, việc tìm kiếm một ứng dụng nghe nhạc phù hợp với nhu cầu cá nhân không phải lúc nào cũng dễ dàng.

Chính vì vậy, chúng em đã chọn đề tài "Phát triển ứng dụng nghe nhạc sử dụng ngôn ngữ Python". Mục tiêu của chúng em là tạo ra một ứng dụng nghe nhạc đơn giản nhưng đầy đủ tính năng, dễ sử dụng và có thể tùy chỉnh theo nhu cầu của người dùng. Chúng em tin rằng, với sự linh hoạt và mạnh mẽ của Python, chúng em có thể đạt được mục tiêu này.

% \begin{mdframed}[hidealllines=true,backgroundcolor=magenta!10]
% 	\begin{lstlisting}
% 		% ------------------------------- %
% 		%     XOA MAN HINH VA CAC BIEN    %
% 		% ------------------------------- %
% 		clear
% 		clc

% 		% ------------------------------- %
% 		%      NHAP DU LIEU BAI TOAN      %
% 		% ------------------------------- %
% 		n = ...;      % So nguoi dan
% 		m = ...;      % So benh vien
% 		% Ma tran D bieu dien thu tu uu tien cua benh vien doi voi benh nhan
% 		% ung voi tung hang
% 		D = [...];
% 		% Ma tran B bieu dien thu tu uu tien cua benh nhan doi voi benh vien
% 		% ung voi tung cot
% 		B = [...];
% 		% Ma tran c bieu dien suc chua cua tung benh vien
% 		c = [...];
% 		% Ma tran a bieu dien moi benh nhan chi duoc chon lua mot benh vien
% 		a = ones(n,1);

% 		% ------------------------------- %
% 		% GIAI BAI TOAN BANG SOLVER MOSEK %
% 		% ------------------------------- %
% 		cvx_begin
% 			cvx_solver mosek
% 			% Bien x(i,j) chi nhan gia tri 0 hoac 1
% 			% ung voi su ghep goi benh nhan r_i voi benh vien h_j
% 			variable x(n,m) binary
% 			% Toi da tong cac bien x(i,j)
% 			% tuc la cang nhieu cap duoc ghep doi cang tot
% 			maximize( 0 )
% 			subject to
% 				% Tong cac hang trong cung mot cot (so benh nhan duoc chon)
% 				% nho hon hoac bang suc chua cua benh vien
% 				sum(x,1) <= c;
% 				% Tong cac cot trong cung mot cot (so benh vien duoc chon)
% 				% nho hon hoac bang 1
% 				sum(x,2) <= a;
% 			% Bao dam khong co cac cap chan
% 			for u = 1:n
% 				for v = 1:m
% 					%Tinh so hang dau tien trong ham dieu kien on dinh
% 					t1 = 0;
% 					for j = 1:m 
% 						t1 = t1 + lt(D(u,j),D(u,v)) * x(u,j); 
% 					end
% 					%Tinh so hang thu hai trong ham dieu kien on dinh
% 					t2 = 0;
% 					for i = 1:n
% 						t2 = t2 + lt(B(i,v),B(u,v)) * x(i,v) / c(v); 
% 					end
% 					%Xac lap ham dieu kien on dinh
% 					t1 + t2 + x(u,v) >= 1;
% 					%Ham dam bao cac cap (r_u,h_v) duoc xet nam trong A, neu
% 					%cap do khong nam trong A thi x_uv = 0
% 					if D(u,v) == m+n+1 || B(u,v) == m+n+1
% 						(eq(D(u,v),m+n+1) + eq(B(u,v),m+n+1)) * x(u,v) == 0;
% 					end
% 				end
% 			end
% 		cvx_end

% 		% ------------------------------- %
% 		%  HIEN THI KET QUA RA MAN HINH   %
% 		% ------------------------------- %
% 		D
% 		B
% 		c
% 		x       % Cac cap duoc ghep doi
% 	\end{lstlisting}
% \end{mdframed}
\subsection{Mục đích - mục tiêu của đề tài}
\textbf{-Mục đích}
\begin{itemize}
	\item Nắm chắc được kỹ năng và kiến thức về ngôn ngữ lập trình Python.
	\item Tìm hiểu về cách thức hoạt động của một ứng dụng nghe nhạc.
	\item Tìm hiểu về thư viện Pygame, MySQL Connector, \dots
	\item Cũng cố, áp dụng, nâng cao kiến thức đã học.
\end{itemize}
\textbf{-Mục tiêu}
\begin{itemize}
	\item Vận dụng được tính chất của lập trình hướng đối tượng.
	\item Xây dựng một ứng dụng nghe nhạc đơn giản, dễ sử dụng.
\end{itemize}
\subsection{Phạm vi đề tài}
\begin{itemize}
	\item Ứng dụng có thể tìm kiếm, phát nhạc từ cơ sở dữ liệu.
	\item Ứng dụng có thể tạo danh sách phát, tạo playlist.
\end{itemize}
\subsection{Nội dung đề tài}
\begin{flushleft}
	Bao gồm 2 phần:
	\begin{itemize}
		\item Phần 1: Mở đầu
		\item Phần 2: Thực hiện ứng dụng nghe nhạc trên Python
		      \begin{itemize}
			      \item Mở đầu
			      \item Xây dựng ứng dụng nghe nhạc bằng cách sử dụng các thư viện của Python:
			            \begin{itemize}
				            \item Phân tích yêu cầu
				            \item Thiết kế kiến trúc
				            \item Thiết kế cơ sở dữ liệu
				            \item Xây dựng giao diện
				            \item Xây dựng chức năng
				            \item Kiểm thử và sửa lỗi
			            \end{itemize}
		      \end{itemize}
	\end{itemize}
\end{flushleft}

\section{Xây dựng ứng dụng nghe nhạc trên Python bằng các thư viện của Python}
\subsection{Đôi nét về ứng dụng nghe nhạc}
\begin{flushleft}
	Ứng dụng nghe nhạc là một ứng dụng giúp người dùng có thể tìm kiếm, phát nhạc từ cơ sở dữ liệu, tạo danh sách phát, tạo playlist, \dots

\end{flushleft}
\subsection{Vẽ biều đồ tần xuất tương đối tích lũy số sinh viên theo phân nhóm outcome các sinh viên có câu trả lời sai trong kỳ thi cuối kỳ trong các nhóm}
\begin{itemize}
	\item \large\bfseries\underline {Phần code R} \newline
\end{itemize}

\newpage
%%%%%%%%%%%%%%%%%%%%%%%%%%%%%%%%%
\begin{thebibliography}{80}

	\bibitem{CVX}
	CVX Introduction
	``\textbf{link: http://cvxr.com/cvx/doc/intro.html/}'',
	\textit{What is CVX}, lần truy cập cuối: 15/04/2017.

\end{thebibliography}
\end{document}

